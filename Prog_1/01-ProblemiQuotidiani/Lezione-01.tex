\documentclass[a4paper]{article}

\usepackage[utf8]{inputenc}
\usepackage[T1]{fontenc}
\usepackage[italian]{babel}
\usepackage{titling}
\usepackage{geometry}

\makeatletter
\newcommand{\subtitle}[1]{\gdef\@subtitle{#1}}
\newcommand{\@subtitle}{}
\pretitle{
  \begin{center}
  \LARGE
}
\posttitle{
  \par\vskip0.5em
  \large\@subtitle
  \end{center}
}
\makeatother

\title{Tutorato 01}
\subtitle{Problemi quotidiani}
\author{Prigione Luca}
\date{09/10/XXXX}

\begin{document}
\pagestyle{empty}

\maketitle

\section*{Note}
    \begin{itemize}
        \item Potete usare solo la libreria standard di input/output quindi \texttt{stdio.h} in C oppure \texttt{iostream} in C++.
    \end{itemize}

\section*{Il mendicante}
Il sole moriva oltre le colline quando Lele raggiunse la locanda. L'insegna di legno pendeva storta, graffiata dal tempo e dal vento.
Appena mise piede sul gradino di pietra, una figura cenciosa emerse dall’ombra accanto alla porta: un vecchio, la barba aggrovigliata come radici e le mani tremanti tese verso di lui.

«Una moneta, buon signore… ché oggi non ho né pane né speranza.»\\
Lele si arrestò un istante. Aveva una gran voglia di bere qualcosa, ma gli restava solo una moneta d’oro. Non poteva certo sprecarla per un mendicante.
Così escogitò un piano.\\
Lele ha due tasche: una contiene la moneta ("c", per *coin*), l’altra è vuota (" ", spazio). Il suo obiettivo è scambiare il contenuto delle due tasche, così da far sembrare quella con la moneta "vuota" — senza mai tirare fuori la moneta o usare una terza tasca.

Dovete quindi scrivere un programma che: dati in input due caratteri " " (spazio) e "c" (coin), li memorizza in due variabili e successivamente ne scambia i valori al proprio interno senza creare alcuna variabile esterna di appoggio.

\section*{Un barista poco studiato}
Nonostante il piano ingegnoso, il mendicante non si lasciò ingannare, in un lampo, afferrò la moneta d’oro e sparì con una risata sdentata.\\
Senza più monete, ma con la gola sempre più secca, Lele entrò ugualmente nella locanda, deciso a tentare la sorte.
Si avvicinò al bancone, dove il locandiere HaFinito lo accolse con un grugnito.

«Una birra, per favore... pagherò domani.»\\
Il locandiere alzò lo sguardo. «Domani? Mmm... vediamo. Se oggi me ne prendi N dello stesso tipo con siascuna prezzo M, quale sarà il costo in totale?»\\
Ci mise un po’ a contare sulle dita, confondendo le moltiplicazioni con le somme.
Lele capì che avrebbe dovuto dargli una mano.

Scrivete un programma che, dati in input due numeri interi positivi `m` e `n`, stampi le moltiplicazioni della tabellina del numero `m` dalla prima fino alla `n`-esima.

\section*{Una cassa mooolto vecchia}
Ottimo lavoro! Grazie al vostro aiuto, il locandiere HaFinito è riuscito a completare la tabellina senza confondersi troppe volte. Soddisfatto, batte le mani e vi serve la birra con un sorriso sdentato.

«Adesso devo solo segnare quanto mi devi sulla mia... ehm... modernissima cassa!»,\\
«Però c'è un problema,» prosegue, grattandosi la testa. «La mia cassa... prende come input i numeri solo in binario.»
Lele sospira. Ormai ci ha fatto l’abitudine. Tocca ancora una volta a voi aiutarlo.

Scrivete un programma che, dato in input un numero intero positivo `n`, stampi la sua rappresentazione in binario.\\
\\
Sfida: Se avete voglia di una piccola sfida provate a salvare il risultato in un'array e poi a stampare gli elementi all'interno di esso

\section*{Il simulatore}
Finita la birra e con l’umore leggermente sollevato, Lele si alzò dal bancone. Ma prima che potesse raggiungere l’uscita, sentì una voce familiare provenire da un angolo buio della locanda.

«No, no! Non può essere un altro uno!»\\
Era il mendicante di prima: Goghi, come lo chiamavano tutti, che se ne stava curvo su un tavolo traballante, le mani nei capelli e una piccola folla di avventori intorno che ridevano sguaiatamente. Davanti a lui, un paio di dadi e alcune monete (tra cui, neanche a dirlo, quella d’oro di Lele).
Avvicinandosi, Lele notò che Goghi stava perdendo tutto al gioco dei dadi, chiaramente truccato.

«Se solo potessi prevedere il prossimo tiro…» mormorò Goghi tra i denti.\\
Lele ci pensò un attimo. Prevedere il futuro, no. Ma simulare un lancio di dadi, quello sì!

Scrivete un programma che simuli il lancio di due dadi nel seguente modo: all'interno della funzione deve essere presente la funzione Lancio che ritorna sotto forma di integer il risultato del lancio di un classico dado da 6\\
\\
Consigli: per includere la libreria rand in C dovete usare inlcudere: \#include <stdlib.h> e \#include <time.h>\\
Consigli: per includere la libreria rand in C++ dovete usare inlcudere: \#include <cstdlib> e \#include <ctime>\\
\\
Sfida: Se avete voglia di una piccola sfida provate a cambiare la dimensione del dado prendendola da input

\section*{Il messaggio segreto}
Goghi, ormai euforico per le sue recenti vittorie, alzò il boccale in un brindisi solitario. Poi si chinò verso Lele e sussurrò, con tono da cospiratore:

«Ti devo qualcosa, amico mio. C’è una cosa che pochi sanno... Fuori da questa locanda, scolpito su una pietra vicino al vecchio pozzo, c’è un messaggio. Parlano di un tesoro, ma… è pieno di simboli strani. Lettere, numeri… antichi.»\\
Lele uscì e trovò la pietra. Era tutto scritto chiaramente… tranne che per quei numeri romani, incisi in fila, come enigmi da decifrare.
Fortunatamente per lui, leggere quei numeri era semplice. Per voi, invece, resta un’ultima sfida prima di partire per l’avventura.

Come ultima sfida dovete scrivere un programma che: dato un'array di char che compone il numero voi dovete riuscire a trasformare tale array in un numero intero.\\
\\
Sfida: Provate ad utilizzare una funzione per effettuare tale Conversione
Super sfida: provate a controllare anche se la stringa che viene passata è composta nel modo corretto (es: non ci sono più di 3 ripetizioni di un carattere,...)

\noindent\rule{\linewidth}{0.4pt}
\section*{Esercizi aggiuntivi}
Se avete già finito e volete qualche esercizio in più chiedete al Tutor, così potete provare a testare il vostro livello.

\end{document}
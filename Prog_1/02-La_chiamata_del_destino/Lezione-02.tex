\documentclass[a4paper]{article}

\usepackage[utf8]{inputenc}
\usepackage[T1]{fontenc}
\usepackage[italian]{babel}
\usepackage{titling}
\usepackage{geometry}
\usepackage{amsmath}

\makeatletter
\newcommand{\subtitle}[1]{\gdef\@subtitle{#1}}
\newcommand{\@subtitle}{}
\pretitle{
  \begin{center}
  \LARGE
}
\posttitle{
  \par\vskip0.5em
  \large\@subtitle
  \end{center}
}
\makeatother

\title{Tutorato 02}
\subtitle{La chiamata del destino}
\author{Prigione Luca}
\date{16/10/XXXX}

\begin{document}
\pagestyle{empty}

\maketitle

\section*{Note}
  \begin{itemize}
    \item Potete usare solo la libreria standard di input/output quindi \texttt{stdio.h} in C oppure \texttt{iostream} in C++.
  \end{itemize}

\section*{La bancarella}
Al mattino, Lele si svegliò deciso: quel messaggio inciso sulla pietra parlava di un tesoro nascosto, e lui doveva prepararsi per l'avventura. Prima di partire, però, serviva l'armamentario giusto.
Si diresse al mercato cittadino, un trambusto di colori, suoni e profumi. Tra le numerose bancarelle, ne trovò una piena di oggetti curiosi.
Il mercante, un uomo basso e robusto con un sorriso furbo, si presentò:

«Io sono Fad. Sei qui per partire? Di sicuro ho quello che fa per te!»
Poi aggiunse: «Oggi c'è un'offerta speciale: se riesci a decifrare qual è l'oggetto che costa meno, ti regalo tutto l'armamentario che vuoi!»\\
Fad osservò Lele con un sorriso malizioso. «Per scoprire qual è l'oggetto che costa meno, dovrai risolvere questo indovinello.»

Sia data una matrice M di dimensioni NxN contenente dei prezzi parziali.
Il vostro compito è calcolare il costo totale di ogni oggetto, sapendo che il costo dell'oggetto in posizione (X,Y) è dato dalla formula:
\begin{equation*}C[X][Y]=M[X][Y]+M[Y][X]\end{equation*}
Successivamente, dovrete individuare il costo minimo tra tutti gli oggetti e stampare una delle coppie di indici (X,Y) che permette di raggiungere tale costo.\\
Lele accettò l'offerta, pronto a mettere alla prova la sua mente prima di scegliere il suo armamentario.

\section*{Imprevisto}
Vista l'astuzia e la bravura di Lele nel risolvere l'enigma, Fad iniziò a preoccuparsi.\\
Proprio quando Lele stava per terminare, il mercante fece un gesto improvviso: raccolse di fretta la sua mercanzia, afferrò il computer di Lele e, senza dire una parola, si dileguò tra la folla del mercato, sparendo a gambe levate.
Poichè il ladro vi ha rubato il computer non potrete più usare l'interfaccia grafica per creare i file, ma dovrete utilizzare il terminale.

\section*{Un gendarme sfaticato}
Dopo il furto, Lele non perse tempo e si diresse subito dalla stazione dei gendarmi, sperando di ottenere aiuto per ritrovare il mercante Fad.
Davanti al banco, incontrò Piterballs, un poliziotto dal carattere pigro e deciso a continuare il suo pisolino pomeridiano. Appena Lele iniziò a raccontare la sua disavventura, Piterballs gli porse un fascicolo pieno di scartoffie da compilare.
«Compila questo, poi vedremo,» brontolò il gendarme, cercando di sbarazzarsi in fretta del “seccatore”.
Lele, esasperato ma determinato, prese in mano le carte, pronto a districarsi tra moduli e richieste burocratiche.
\begin{equation*}void strcpy(char *src, char *dest)\end{equation*}
Dovete scrivere una funzione:  che dato un vettore di caratteri in input `src' copia il contenuto in `dest' ed una funzione: \begin{equation*}bool strcmp(char *src, char *dest)\end{equation*} che dati i vettore di caratteri in input `src' e `dest' ritorna se i due vettori sono uguali.\\
P.S.: Inutile dire che non potete usare le funzioni già esistenti presenti nella libreria standard.\\
\\
Sfida: Per i ragazzi che hanno fatto le stringhe potete provare a riscrivere la funzione utilizzando le strighe al posto dei vettori di caratteri.

\section*{Banca dati}
Una volta compilati tutti i moduli con pazienza, precisione e l'aiuto di qualche funzione ben scritta Lele tornò dal gendarme, portando con sé una pila ordinata di scartoffie.
Piterballs, che ormai non aveva più scuse per continuare a dormire, sospirò rumorosamente.

«Ugh… adesso mi tocca ricordare a memoria tutte le persone con cui dovrò parlare… e le loro deposizioni!» mugugnò, già sudando al pensiero.\\
Ma prima che potesse disperarsi troppo, Lele fece un passo avanti.

«Tranquillo, posso aiutarti io. Costruiremo una banca dati semplice ma efficace così da non poter dimenticare più nulla.»\\
Il volto di Piterballs si rallegrò all'improvviso, come se una luce si fosse accesa nella sua mente stanca, il gendarme spostò un mucchio di carte da una sedia e fece cenno a Lele di accomodarsi.
Era tempo di costruire insieme una vera banca dati.

Scrivete un programma che simuli una banca dati nella quale sia possibile gestire numeri di identificazione.
Il programma dovrà avere 4 funzioni che permetteranno di:
\begin{itemize}
  \item \textbf{aggiungere} un nuovo numero alla banca dati,
  \item \textbf{rimuovere} un numero esistente,
  \item \textbf{cercare} un numero per verificarne la presenza,
  \item \textbf{stampare} tutti i numeri attualmente presenti nella banca dati.
\end{itemize}
Sfida: Per i ragazzi che hanno nozioni di complessità potete provare a implementare la banca dati con una ricerca di costo O\ (log(n)).

\noindent\rule{\linewidth}{0.4pt}
\section*{Esercizi aggiuntivi}
Se avete già finito e volete qualche esercizio in più chiedete al Tutor, così potete provare a testare il vostro livello.

\end{document}
\documentclass[a4paper]{article}

\usepackage[utf8]{inputenc}
\usepackage[T1]{fontenc}
\usepackage[italian]{babel}
\usepackage{titling}
\usepackage{geometry}
\usepackage{amsmath}

\makeatletter
\newcommand{\subtitle}[1]{\gdef\@subtitle{#1}}
\newcommand{\@subtitle}{}
\pretitle{
  \begin{center}
  \LARGE
}
\posttitle{
  \par\vskip0.5em
  \large\@subtitle{}
  \end{center}
}
\makeatother

\title{Tutorato 04}
\subtitle{Fuori dalle mura}
\author{Prigione Luca}
\date{06/11/XXXX}

\begin{document}
\pagestyle{empty}

\maketitle

\section*{Note}
  \begin{itemize}
    \item Potete usare solo la libreria standard di input/output quindi \texttt{stdio.h} in C oppure \texttt{iostream} in C++.
  \end{itemize}

\section*{Il frate}
Lasciata alle spalle la confusione dei bassifondi, Lele si avviò finalmente verso il limitare della città, l’avventura lo attendeva.
Camminando, intravide una figura familiare sul sentiero.
Era il Frate Gab.
Da anni viveva nel monastero cittadino, ma tutti sapevano che il suo sogno era quello di viaggiare.
Quando si accorse di Lele, il frate gli rivolse un sorriso luminoso:

«Fratello viandante! Che lieta coincidenza! Anch’io mi sto incamminando verso le mura. Vuoi condividere un tratto di strada?»\\
Dopo un po’, per passare il tempo, Gab propose con entusiasmo:

«Che ne dici di un piccolo gioco? È un indovinello molto antico tra i frati»

«Il gioco è semplice, ti verranno date due cifre sotto forma di numero decimale e tu dovrai scambiare tra i due numeri le corrispondenti ultime 8 cifre della propria rappresentazione binaria. Insomma, in fin dei conti è solo semplice matematica!»\\
Lele sorrise. Di giochi così ne aveva già affrontati, ma non voleva deludere l’amico.\\
P.S.: Per chi lo sapesse fare dovete prendere i numeri in imput da linea di comando

\section*{La guardia}
Risolto l’indovinello, Lele e Frate Gab raggiunsero finalmente le imponenti mura della città.
Quando arrivarono di fronte al grande portone di quercia scura, una guardia li fermò con un gesto brusco.

«Fermi lì. Per uscire da queste mura non basta il desiderio di viaggiare,» disse con tono severo.\\
«Serve dimostrare abilità e logica. Altrimenti, nessuno passa.»
La guardia si piazzò davanti al portone, incrociando le braccia.

«Vi darò una lista di numeri, alcuni positivi e altri negativi. Il vostro compito è questo:
trovare la somma più alta ottenibile scegliendo solo numeri consecutivi all’interno della lista.»\\
Fece un sorriso appena accennato, come se già pregustasse il fallimento dei due viaggiatori.\\
Quando i due conclusero, la guardia controllò il risultato e annuì.

«Avete risolto la prova… ma non potete comunque lasciare la città.
Serve un permesso del Consiglio. Ordini.»

\section*{Le mura}
Lele e Gab, delusi, si prepararono a tornare indietro.
Fu allora che una voce bassa arrivò da un vicolo vicino:

«Non prendetevela. Quella guardia non fa altro che ripetere le regole.»\\
Dal muro alle loro spalle si staccò una figura agile, con un mantello chiaro e uno zaino da viaggio.

«Il mio nome è Daniele, esploratore e cartografo.»

«Se davvero volete uscire,» disse abbassando la voce, «esiste un modo.»\\
Daniele aprì lentamente una mappa sgualcita: una pianta intricata delle mura.\\
Tra i bastioni, i corridoi e le fondamenta, si vedevano sottili passaggi secondari, non riportati nelle mappe ufficiali, né conosciuti dalle guardie.

«Questa è la mappa dei passaggi segreti. Non è perfetta… molte parti sono consumate. Per questo serve ingegno.»

Dovete creare un programma che data una mappa di dimensione NxN contenete degli 0 (Passaggio) e degli 1 (Muro) restituisca se, partendo dalla loro posizione di partenza (la cella (0,0)), si può raggiungere o meno la libertà (la cella (N,N)). Ci si può muovere in verticale ed in orizzontale, ma non in diagonale\\
Nel caso in cui sia possibile raggiungere la libertà stapate un possibile passaggio per i futuri avventurieri.

\noindent\rule{\linewidth}{0.4pt}
\section*{Esercizi aggiuntivi}
Se avete già finito e volete qualche esercizio in più chiedete al Tutor, così potete provare a testare il vostro livello.

\end{document}
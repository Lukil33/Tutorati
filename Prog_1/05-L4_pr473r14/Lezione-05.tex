\documentclass[a4paper]{article}

\usepackage[utf8]{inputenc}
\usepackage[T1]{fontenc}
\usepackage[italian]{babel}
\usepackage{titling}
\usepackage{geometry}
\usepackage{amsmath}

\makeatletter
\newcommand{\subtitle}[1]{\gdef\@subtitle{#1}}
\newcommand{\@subtitle}{}
\pretitle{
  \begin{center}
  \LARGE
}
\posttitle{
  \par\vskip0.5em
  \large\@subtitle
  \end{center}
}
\makeatother

\title{Tutorato 05}
\subtitle{L4 pr473r14}
\author{Prigione Luca}
\date{13/11/XXXX}

\begin{document}
\pagestyle{empty}

\maketitle

\section*{Parole distorte}
Superate finalmente le mura, Lele, Frate Gab e Daniele imboccarono la strada che si allontanava dalla città, il mondo oltre le mura sembrava più vasto di quanto avessero mai immaginato.\\
Dopo qualche ora di cammino, Daniele si fermò all’incrocio di un sentiero.

«Da qui le nostre strade si dividono,» disse, osservando l’orizzonte.

«Devo continuare verso nord, verso le montagne. Voi proseguite a ovest, verso la pianura.»\\
Lele lo salutò con rispetto, mentre Frate Gab mormorò una benedizione.\\
Proseguendo il viaggio, i due si imbatterono in una grande pietra, coperta a metà dal muschio e dall'erba alta.
Sulla superficie, incisi con cura, si distinguevano strani simboli, un intreccio di lettere e numeri.
Frate Gab si chinò per osservarla meglio.

«Pare un messaggio antico… ma alcuni caratteri non sono corretti. Forse quei numeri celano lettere dimenticate.»

Scrivete un programma che, data una sequenza di caratteri (lettere e numeri) memorizzata in un array, ricostruisca ricorsivamente la sequenza sostituendo ogni numero con la lettera più simile.\\
P.S.: Per chi ha fatto la lettura da linea di comando, deve richiedere l'array da lì.

\section*{Le due sorelle}
Superata la pietra incisa, Lele e Frate Gab proseguirono lungo la strada sterrata, finché il paesaggio cambiò:
davanti a loro si apriva una distesa di mais dorato, che ondeggiava al vento come un mare luminoso.
Camminarono per un tratto, finché risate leggere ruppero il silenzio.
Tra le spighe, due bambine giocavano lanciandosi dei sassolini numerati e spostandoli in modo curioso.

«Mi chiamo Giad,» disse la più grande, con un sorriso furbo.

«E io sono Ale!» aggiunse l’altra, saltellando.

«Questo è il nostro gioco segreto. Volete provarci anche voi?»\\
Lele esitò, ma Frate Gab fece un passo avanti.

«Perché no? Ci piacciono le sfide.»\\
Dovete creare un programma che, data una sequenza di numeri memorizzata in un array, elimini in modo ricorsivo tutti i numeri il quale valore è maggiore o uguale alla somma di tutti i successivi valori validi.\\
P.S.: Un numero viene considerato valido solo se non è stato eliminato in precedenza.\\
P.P.S.: Tutti i numeri sono $\geq$ 0.

\section*{Ancora numeri binari???}
Risolto l’enigma delle sorelle, Lele e Frate Gab sorrisero soddisfatti.

«Siete state brave» disse il frate con tono bonario, «ma ora tocca a me proporre un gioco.
È un piccolo indovinello che i monaci del mio convento amavano risolvere durante le sere d’inverno.»\\
Le due sorelle si sporgono, incuriosite.
«Un altro gioco? Quale?» chiese Giad.\\
Frate Gab prese un bastoncino e disegnò due file di cifre sulla terra.
«Riuscireste a sommare due numeri binari… in modo ricorsivo?»\\
Le bambine si guardarono perplesse, mentre Ale sussurrò: «Binari? Come quelli del treno?»\\
Lele rise piano. «No, piccole. Questa volta si parla di numeri.»

Dovete scrivere un programma che, dati due numeri binari memorizzati in due array, ne calcoli la somma memorizzandola in un terzo array, il tutto in modo ricorsivo.
P.S.: I numeri binari sono memorizzati in ordine inverso, cioè la cifra meno significativa è all'inizio dell'array.
P.P.S.: Per chi ha fatto la lettura da linea di comando, deve richiedere l'array da lì.

\section*{La spirale di grano}
Dopo aver spiegato alle sorelle come risolvere l’enigma dei numeri binari, Lele e Frate Gab ripresero il cammino.
Procedevano in silenzio, quando Lele si fermò di colpo.

«Frate… guarda laggiù.»\\
Nel mezzo della distesa dorata, il grano sembrava piegato in modo innaturale, formando una spirale perfetta.
Frate Gab socchiuse gli occhi.

«Non è opera del caso. Qualcuno ha voluto lasciare un segno… un percorso da seguire.»\\
Lele si chinò, tracciando il disegno con il dito.

Immagina una griglia NxN in cui ogni casella contiene un numero. La casella in alto a sinistra contiene sempre 1 e, procedendo a spirale in senso orario a partire da quella, ogni casella conterrà il numero naturale successivo.\\
Il tuo compito è quello di scoprire data in input la dimensione della matrice e data una colonna n quant'è la somma dei numeri presenti all'interno di quella colonna.\\
P.S.: L'esercizio è molto difficile, quindi non preoccupatevi se non riuscite a risolverlo in modo ricorsivo.

\noindent\rule{\linewidth}{0.4pt}
\section*{Esercizi aggiuntivi}
Se avete già finito e volete qualche esercizio in più chiedete al Tutor, così potete provare a testare il vostro livello.

\end{document}
\documentclass[a4paper]{article}

\usepackage[utf8]{inputenc}
\usepackage[T1]{fontenc}
\usepackage[italian]{babel}
\usepackage{titling}
\usepackage{geometry}
\usepackage{amsmath}

\makeatletter
\newcommand{\subtitle}[1]{\gdef\@subtitle{#1}}
\newcommand{\@subtitle}{}
\pretitle{
  \begin{center}
  \LARGE
}
\posttitle{
  \par\vskip0.5em
  \large\@subtitle{}
  \end{center}
}
\makeatother

\title{Tutorato 06}
\subtitle{Il monastero}
\author{Prigione Luca}
\date{20/11/XXXX}

\begin{document}
\pagestyle{empty}

\maketitle

\section*{Il Monastero del nido}
Il sentiero si arrampicava sulle colline finché, tra gli alberi, apparve il Monastero del Nido: una costruzione chiara, raccolta e silenziosa come un rifugio d’aquila.
Appena arrivati al portone, questo si aprì di scatto.

«Gab! Finalmente!»\\
Un frate snello e sorridente li accolse con un abbraccio.

«Benvenuto, straniero. Io sono Frate Ballocco, colui che gestisce il monastero.
Prima di entrare, però, dovrò sottoporti alla prova degli apprendisti.»\\
Lele si irrigidì.

«Una prova?»\\
Ballocco annuì con aria soddisfatta.\\
Dovrai creare una struct che rappresenti un monaco, con i seguenti campi:
\begin{itemize}
    \item Nome
    \item Età
    \item Mansione
    \item Anni di esperienza
\end{itemize}
Successivamente dovrai permettere di eseguire le segunti operazioni:
\begin{enumerate}
    \item Creare un array di N monaci con valori a piacere
    \item Scrivere una funzione che stampi i dati di tutti i monaci
    \item Scrivere una funzione che ordini i monaci in base agli anni di esperienza 
\end{enumerate}
P.S.: La funzione per ordinare deve funzionare da toggle, cioè se i monaci sono ordinati in ordine crescente deve ordinare in ordine decrescente e viceversa.

\section*{Il passo iTAGLIco}
Entrati nel monastero, Frate Ballocco condusse Lele e Frate Gab in una piccola sala piena di volumi antichi.

«Questi sono dei testi scritti in iTAGLIano,» spiegò. «Una lingua curiosa: tutte le parole scritte \textbf{interamente} in maiuscolo sono… al contrario. Per leggerle serve invertirle.»\\
Ballocco posò un manoscritto sul tavolo.

«Hai mostrato un buon occhio. Ti andrebbe di aiutarci con una traduzione?»\\
Frate Ballocco chiuse il volume con un tonfo leggero.

«Se riuscirai a farlo,» disse, «le porte della nostra biblioteca saranno aperte per te quanto vorrai.»\\
Lele annuì, pronto a mettersi al lavoro tra i silenzi antichi del Monastero del Nido.
Scrivete un programma che:
\begin{itemize}
    \item Legga da linea di comando il nome dei file necessari.
    \item Legga il file passato a riga di comando parola per parola.
    \item Modifichi il testo trasformandolo in italiano.
    \item Riscriva il testo tradotto all'interno del file passato a riga di comando.
\end{itemize}
P.S.: Se non sapete leggere da riga di comando potete chiamare i file input.txt e output.txt

\section*{La biblioteca interrata}
Frate Ballocco, impressionato dall’abilità di Lele, lo condusse lungo una scala a chiocciola che scendeva sotto il monastero.

«Qui conserviamo ciò che non deve andare perduto,» spiegò mentre la torcia proiettava ombre sulle pareti.

«Tuffu!» chiamò Ballocco. «Ho qualcuno che potrà aiutarti.»\\
La giovane amanuense si voltò di scatto. Aveva gli occhi svegli, le mani macchiate d’inchiostro e un mucchio di fogli sotto il braccio.

«Tu devi essere Lele,» disse subito, come se lo stesse aspettando. «Ho già sentito parlare delle tue capacità.
Perfetto! Allora puoi aiutarmi a catalogare questi libri… prima che mi crollino tutti addosso.»\\
Lele guardò la montagna di tomi e sorrise debolmente.
Dovrai scrivere un programma che:
\begin{enumerate}
    \item Legga da un file di testo una lista di libri, uno per riga.
    \item Per ogni libro dovrà leggere il file ``nome\_libro.txt'' per ottenere maggiori informazioni
    \item Dopodichè bisognerà creare una struct libro che contiene:
    \begin{itemize}
        \item Nome
        \item Autore
        \item Anno di pubblicazione
        \item Numero di caratteri (Spazi esclusi)
    \end{itemize}
    \item Infine bisognerà stampare a schermo la lista dei libri
\end{enumerate}
P.S.: Se non sapete leggere da riga di comando potete chiamare il file input.txt

\section*{Il testo Nascosto}
Dopo ore passate ad aiutare Tuffu a rimettere ordine tra registri, pergamene e tomi dimenticati, l’amanuense si fermò finalmente… e sorrise.

«Lele, senza di te sarei ancora sommersa fino al collo, per ringraziarti voglio mostrarti qualcosa che nessuno ha il permesso di leggere.»\\
Da uno scaffale nascosto estrasse un volume pesante con il titolo inciso a mano:
\begin{center}
    “Deficienza Artificiale” \- M.Romberi
\end{center}
«È il libro più importante che possiedo,» disse Tuffu con voce bassa.\\
«Un testo enigmatico che parla di un tesoro perduto… ma è scritto talmente male che nessuno è mai riuscito a decifrarlo del tutto.»\\
Gli porse il tomo.

«Se davvero vuoi provarci, ti affido un compito degno della tua fama.»
Per decifrarlo ti posso dare i seguenti suggerimenti:
\begin{enumerate}
    \item Oltre al testo.txt ci sono anche due fogli: parole\_criptate.txt, parole\_decriptate.txt.
    \item Ogni parola i presente sul foglio parole\_criptate.txt deve essere tradotta con la parola i presente sul foglio parole\_decriptate.txt.
    \item Il testo è case sensitive, quindi Bolla, BOlla e bolla sono parole differenti.
    \item Devi stare infine attento perchè il testo potrebbe necessitare di essere tradotto più volte.
\end{enumerate}

\noindent\rule{\linewidth}{0.4pt}
\section*{Esercizi aggiuntivi}
Se avete già finito e volete qualche esercizio in più chiedete al Tutor, così potete provare a testare il vostro livello.

\end{document}
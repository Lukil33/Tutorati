\documentclass[a4paper]{article}

\usepackage[utf8]{inputenc}
\usepackage[T1]{fontenc}
\usepackage[italian]{babel}
\usepackage{titling}
\usepackage{geometry}
\usepackage{amsmath}

\makeatletter
\newcommand{\subtitle}[1]{\gdef\@subtitle{#1}}
\newcommand{\@subtitle}{}
\pretitle{
  \begin{center}
  \LARGE
}
\posttitle{
  \par\vskip0.5em
  \large\@subtitle{}
  \end{center}
}
\makeatother

\title{Tutorato 07}
\subtitle{Le sacerdotesse del pozzo}
\author{Prigione Luca}
\date{27/11/XXXX}

\begin{document}
\pagestyle{empty}

\maketitle

\section*{Le sacerdotesse}
Dopo aver chiuso il libro di M. Roberi, Tuffu guidò Lele fuori dal monastero fino a un piccolo santuario costruito attorno a un pozzo antico.\\
Prima ancora di arrivare, si sentirono due voci discutere animatamente.

«Sprechi spazio inutilmente!»

«E tu lo lasci collassare!»\\
Tuffu sospirò.\\
Tuffu gli rivolse un sorriso rassegnato.

«Credo che tu sia l’unico in grado di sistemare la questione.»\\
Dovete creare un programma per gestire il pozzo implemendo un vettore dinamico di interi.\\
Ogni volta che da input viene inserito un numero, controllate se è già presente nel vettore: se sì, la dimensione va decrementata dinamicamente; altrimenti va aumentata.\\
Fate però attenzione ai casi limite:\\
Se il pozzo è in secca (0 elementi) oppure sta straripando (10 elementi), stampate il contenuto del pozzo e terminate il programma.

\section*{Un messaggio dal pozzo}
Risolta la disputa tra Alice e Maria, il pozzo tornò silenzioso.
Tuffu si schiarì la voce.

«Non è ancora finita, Lele. Il santuario custodisce anche un messaggio… ma nessuno è mai riuscito a leggerlo per intero.»\\
Dal muro emerse una piccola nicchia con un rotolo polveroso.

«Il testo è frammentato» spiegò Tuffu.

«Ogni riga è sparsa in un file diverso. Per decifrarlo serve ordine… e memoria.»\\
Riceverete i seguenti file passati attraverso riga di comando:
\begin{enumerate}
    \item riceverete il numero di righe e successivamente due numeri per riga:
    \begin{itemize}
        \item il primo numero della riga indica la posizione all'interno del messaggio finale
        \item il secondo numero della riga indica il numero di parole contenute in quella parte di messaggio
    \end{itemize}
    \item riceverete un testo unico contenenti tutte le parti del messaggio in ordine contiguo e date le informazioni precedenti dovrete ricomporre il messaggio.
\end{enumerate}

\section*{La camera dei segreti}
Quando il santuario finalmente si aprì, una corrente d’aria gelida attraversò la stanza.
Una scala scendeva ancora più in profondità.\\
Tuffu abbassò la voce.

«Qui sotto si trova la Camera dei Segreti. Ogni parola viene trasformata… e mai restituita uguale.»\\
Lele si calò al suo interno.
Al centro, un leggio custodiva un altro file avvolto nella polvere.

«Questa volta» disse Tuffu, «non basta leggere.Il testo cambia da solo, parola dopo parola. Serve qualcuno capace di sistemarlo… ricorsivamente.»\\
Scrivete un programma che legga da file di testo una stringa ben formata (ovvero con \texttt{\textbackslash0} finale).
Salvate la stringa all'interno di un vettore allocato dinamicamente e successivamente applicatevi la seguente funzione ricorsiva:
\begin{center}
char* \textbf{Funzione\_Ricorsiva} (\textit{stringa}, \textit{posizione})
\end{center}
La funzione deve restituire una nuova stringa (ben formata) contenente i caratteri '@' estratti dalla stringa originale.\\
P.S.: La dimensione della stringa originale è passata come primo argomento all'interno del file di testo.\\
P.P.S.: Questo è il testo di un es 2 di vecchio esame del Dr.Roveri. 

\section*{L’addestramento}
Terminata la prova del santuario, Lele risalì finalmente dal pozzo.
Ad attenderlo c’era Frate Ballocco, questa volta con un’espressione più pragmatica che solenne.

«Hai fatto progressi notevoli» disse, «ma non sei ancora pronto.
Per affrontare ciò che ti aspetta dovrai imparare a gestire due tecniche fondamentali: ciò che si conserva in cima e ciò che scorre verso il fondo.»\\
Frate Gab annuì.

«Niente misteri questa volta. Solo addestramento.»\\
Implementare un programma che:
\begin{enumerate}
    \item Gestisca una pila (stack) usando un vettore dinamico di interi
    \begin{itemize}
        \item implementa operazioni: push, pop, isEmpty
        \item aumentare la capacità quando pieno
        \item pop deve restituire l’elemento rimosso
    \end{itemize}
    \item Gestisca una coda (queue) usando un vettore dinamico di interi
    \begin{itemize}
        \item implementa operazioni: enqueue, dequeue, isEmpty
        \item aumentare la capacità quando pieno
        \item dequeue deve restituire l’elemento rimosso
    \end{itemize}
\end{enumerate}
P.S.: Se vi sentite proprio coraggiosi potete pensare di implementare il tutto utilizzando le Struct

\noindent\rule{\linewidth}{0.4pt}
\section*{Esercizi aggiuntivi}
Se avete già finito e volete qualche esercizio in più chiedete al Tutor, così potete provare a testare il vostro livello.

\end{document}